\documentclass[12pt]{article}
\usepackage[utf8]{inputenc}
\usepackage[margin=2cm]{geometry}
\usepackage[spanish]{babel}

\newcommand{\shehulk}{\emph{She-HULK} }
\newcommand{\hulk}{\emph{HULK} }
\newcommand{\ini}{$>$ }
\newcommand{\inline}{\textit{inline} }
\newcommand{\letin}{\textit{let-in} }
\newcommand{\ifelse}{\textit{if-else} }

\title{Informe sobre Intérprete \shehulk \emph{: Sheila's Havana University Language for Kompilers}}
\author{Sheila Roque Alemán}
\date{ }

\begin{document}
\maketitle

\section*{Introducción}
	\shehulk es un intérprete sencillo que implementa ciertas características del lenguaje de programación \emph{HULK: Havana University Language for Kompilers}.
	\emph{HULK} es un lenguaje de programación imperativo, funcional, estática y fuertemente tipado, creado y diseñado por la Facultad de Matemática y Computación de la Universidad de La Habana, \emph{MatCom}.
	
\section*{\shehulk}
	El intérprete de \shehulk es una aplicación de consola, donde el usuario puede introducir una expresión de \hulk, presionar [ENTER], e immediatamente se verá el resultado de evaluar expresión (si lo hubiere).
	Cada línea que comienza con '$>$' representa una entrada del usuario, e immediatamente después se imprime el resultado de evaluar esa expresión, si lo hubiere.
	Si el usuario presiona [ENTER] sin haber escrito una instrucción se tomará como una señal de que desea cerrar el programa

	En este intérprete solo se aceptan expresiones de una línea.
	Para ello se han implementado diversas expresiones \inline , como lo son la declaración de funciones de este tipo y ciertas expresiones como las \letin e \ifelse.
	
	A pesar de esta restricción, el intérprete posee la capacidad y optimización necesarias para poder crear un programa completamente funcional y veloz habiéndose creado las funciones necesarias línea por línea.
	
\subsection*{Expresiones básicas}
\begin{itemize}
	\item Todas las instrucciones en \shehulk terminan en "\textit{;}".
	\item La instrucción más simple en \shehulk que hace algo es:
	
	\emph{
	\ini print("Hello, World!"); \\
	Hello, World!
	}
	\item \shehulk además tiene expresiones y funciones aritméticas básicas:
	
	\emph{
	\ini print((((1 + 2) \^\ 3) * 4) / 5); \\
	21.6 \\
	\ini print(sin(2 * PI) \^\ 2 + cos(3 * PI / log(4, 64))); \\
	-1
	}
	\item \shehulk tiene tres tipos básicos: \emph{String}, \emph{Number}, y \emph{Boolean}.
\end{itemize}

\subsection*{Funciones}
\begin{itemize}
	\item En \hulk existen dos tipos de funciones, \inline y regulares.
	En \shehulk solo se implementan las funciones \inline.
	Tienen el formato siguiente:
	
	\emph{\ini function tan(x) $=>$ sin(x) / cos(x);}
	
	\item Una vez definida una función, puede usarse en una expresión cualquiera:
	
	\emph{
	\ini print(tan(PI/2)); \\
	$\infty$
	}
	\item Todas las funciones declaradas anteriormente son visibles en cualquier expresión subsiguiente.
	
	* \emph{Las funciones no pueden redefinirse}.
\end{itemize}
	
\subsection*{Variables}
\begin{itemize}
	\item En \shehulk es posible declarar variables usando la expresión \letin, que posee la siguiente sintaxis:
	
	\emph{
	\ini let x = PI/2 in print(tan(x)); \\
	$\infty$	
	}
	\item Una expresión \letin consta de una o más declaraciones de variables, y un cuerpo, que puede ser cualquier expresión donde además se pueden utilizar las variables declaradas en el \textit{"let"}.
	
	\emph{
	\ini let number = 42, text = "The meaning of life is"\ in print(text @ number); \\
	The meaning of life is 42 \\
	\ini let number = 42 in (let text = "The meaning of life is"\ in (print(text @ number))); \\
	The meaning of life is 42
	}
	\item Fuera de una expresión \letin las variables dejan de existir.
	\item El valor de retorno de una expresión \letin es el valor de retorno del cuerpo, por lo que es posible hacer:
	
	\emph{
	\ini print(7 + (let x = 2 in x * x)); \\
	11
	}
\end{itemize}
	
\subsection*{Condicionales}
\begin{itemize}
	\item Las condiciones en \shehulk se implementan con la expresión \ifelse , que recibe una expresión booleana entre paréntesis, y dos expresiones para el cuerpo del \textit{if} y el \textit{else} respectivamente.
	
	* \emph{Siempre deben incluirse ambas partes}
	
	\emph{
	\ini let a = 42 in if ( a \% 2 == 0 ) print ($"$even$"$) else print ($"$odd$"$); \\
	even
	}
	
	\item Como \ifelse es una expresión, se puede usar dentro de otra expresión
	
	\emph{
	\ini let a = 42 in print ( if ( a \% 2 == 0 ) $"$even$"$ else $"$odd$"$ ); \\
	even
	}
\end{itemize}

\subsection*{Funciones recursivas}
\begin{itemize}
	\item Una función recursiva en \shehulk es la siguiente:
	
	\emph{
	\ini function fib(n) $=>$ if (n $>$ 2) fib(n-1) + fib(n-2) else 1;	\\
	\ini fib(5); \\
	5 \\
	\ini let x = 3 in fib(x+1); \\
	3 \\
	\ini print(fib(6)); \\
	8
	}
\end{itemize}

\subsection*{Errores}
\begin{itemize}
	\item En \shehulk hay 3 tipos de errores que pueden ser detectados: \emph{Léxico}, \emph{Sintáctico} y \emph{Semántico}.
	\item En caso de detectarse un error, el intérprete imprime una línea lo más informativa posible indicando el error.
	\item Al detectar un error, el intérprete detiene la compilación el código y lanza un error, sin causar que el programa caiga, dándole la oportunidad al usuario de corregir su error y continuando con su ejecución.
	\item En caso de ocurrir más de un error, solo se lanza el primero que aparece.
\end{itemize}

\subsubsection*{Errores Léxicos}
	Errores que se producen por la presencia de tokens inválidos o falta de comillas de cierre en los strings:
	
	\emph{\ini let 14a = 5 in print(14a); \\
	! LEXICAL ERROR: '14a' is not a valid token. \\
	\ini print("Hello, World!); \\
	! LEXICAL ERROR: Missing closing $ $ $'$ $"$ $'$ in string expression.}

\subsubsection*{Errores Sintácticos}
	Errores que se producen por expresiones mal formadas como paréntesis no balanceados o expresiones incompletas:
	
	\emph{
	\ini let a = 5 in print(a; \\
	! SYNTAX ERROR: Missing closing parenthesis after 'a'. \\
	\ini let a = 5 inn print(a); \\
	! SYNTAX ERROR: Missing 'in' keyword in 'let-in' expression. \\
	\ini let a = in print(a); \\
	! SYNTAX ERROR: Invalid expression after variable 'a' in 'let-in' expression.
	}

\subsubsection*{Errores Semánticos}
	Errores que se producen por el uso incorrecto de los tipos y argumentos:
	
	\emph{
	\ini let a = $"$Hello, World!$"$ in print(a + 5); \\
	! SEMANTIC ERROR: Operator '+' cannot be used between 'String' and 'Number'. \\
	\ini print(fib("Hello, World!")); \\
	! SEMANTIC ERROR: Function 'fib' receives 'Number', not 'String'. \\
	\ini print(fib(4, 3)); \\
	! SEMANTIC ERROR: Function 'fib' receives 1 argument(s), but 2 were given.
	}

%\section*{¿Cómo funciona?}
%	Al recibir una instrucción por parte del usuario comienza el proceso de compilación.
%	Primero se convierte la instrucción en una lista de tokens en orden de aparición, para facilitar el proceso de \textit{parsing} que le sigue a continuación.
%	Luego del proceso de \textit{parsing} se obtiene como resultado un árbol de sintaxis abstracta compuesto por las distintas expresiones que componen a la instrucción recibida.
%	Luego se realiza la revisión semántica de las expresiones en el árbol de derivación, y luego se evalúan para obtener el valor a devolver (en caso de que lo hubiere).
	
%\subsection*{Tokenizador}
%\begin{itemize}
%	\item El lexer creado tendrá una instancia de \emph{Error Léxico}, la línea de código dada que se va a estar utilizando, y un indizador que nos ayude a recorrerla
%	\item La línea de código la provee la llamada al método \emph{Tokenizer(string line)}
%	\item En caso de que ocurra un error léxico al tokenizar, se lanza dicho error notificando al usuario que el token encontrado no es válido, y se devuelve el token como nulo, para que luego, de existir algún token nulo, se devuelva una lista nula y se detenga la ejecución para empezar desde el principio
%	\item El recorrido del token es sencillo: \\
%	Se recorre toda la línea de código, caracter por caracter, y se va analizando, según el tipo de caracter que sea, qué tipo de token puede ser
%	\begin{itemize}
%		\item Si se encuentra un espacio no se debe guardar nada
%		\item Si se encuentra con unas comillas dobles entonces lo siguiente debe ser una expressión de string. \\
%		Se guarda el string además de las comillas dobles que lo acompañan
%		\item Si se encuentra algún otro caracter válido hay que analizar cúal es y qué tipo de operador devuelve (separadores, operadores, palabraas reservadas, identificadores o literales numéricos), y se devuelve dicho token
%	\end{itemize}
%\end{itemize}

%\subsection*{Parser}
%\begin{itemize}
%	\item El parser posee una instancia de \emph{Error Sintáctico}, una lista de tokens dada y un índice que nos ayuda a recorrer todos los tokens de dicha lista
%	\item El proceso de \textit{parsing} evalúa la lista de tokens dada y devuelve la expressión resultante según la sintaxis
%	\item Se realiza el \textit{parsing} por medio del algoritmo de \textit{parsing recursivo descendiente predictivo}.
%	Consiste en evaluar los tokens en orden de izquierda a derecha, formando las expresiones de forma descendiente, y prediciendo cuál puede ser la posible expresión teniendo en cuenta el primer token a evaluar de esta.	
%\end{itemize}
%	Un pequeño ejemplo pudiera ser la instrucción siguiente:
	
%	\emph{\ini print((1+2)\^\ 3);}
	
%	La lista de tokens dados sería de esta forma:
	
%	print , ( , ( , 1 , + , 2 , ) , \^\ , 3 , ) , ;
	
%	Al empezar a 'parsear', evaluamos qué espresión es la expresión más externa de la instrucción evaluando al primer token.
%	En este caso tenemos a la palabra reservada '\emph{print}', entonces intuímos que la expresión es una llamada al método \emph{print('Expr')}.
	
%	Evaluamos el resto de la expresión teniendo en cuenta la sintaxis de una expresión de llamado al print.
%	Revisamos el siguiente token, procurando que sea un '(', lo cual es cierto en este caso.
	
%	Luego hay que evaluar la expresión interna del print.
%	Primero, nos encontramos que el siguiente token es un '(', lo cual nos indica que existe otra expresión más interna dentro de unos paréntesis.
%	Se evalúa el siguiente token y encontramos que es un literal numérico, y consecuentemente inferimos que la expresión es una expresión numérica.
	
%	Una expresión numérica se evalúa con las siguientes reglas en cuanto al parsing descendiente y predictivo:
	
%	NumExpr -$>$ T X
	
%	T -$>$ num Y $|$ ( NumExpr ) Y $|$ ...
	
%	X -$>$ + NumExpr $|$ - NumExpr $|$ e
	
%	Y -$>$ * T $|$ / T $|$ \% T $|$ \^\ T $|$ e
	
%	Entonces evaluamos el token '1' como el inicio de una expresiónn numérica, que a su vez se va a evaluar como un término, que se evalúa en este caso en la forma T -$>$ num Y, siendo '1' el 'num'.
	
%	Hasta ahora la derivación va siendo: ( num Y X
	
%	Luego evaluamos con la sintáxis de Y.
%	Al ser el siguiente token el operador '+', entonces tenemos que Y -$>$ e, y continuamos hacia el evaluar X.
%	La expresión X resulta en este caso como X -$>$ + NumExpr, y este NumExpr se deriva como NumExpr -$>$ num Y, donde Y -$>$ e.
%	Luego encontramos los paréntesis de cierre de esta expresión
	
%	Ahora tenemos la derivación: (num + num)
	
%	Evaluando que la expresión de externa es otra expresión numérica (que puede ser otro tipo de expresión que pueda utilizar una numérica, pero en este caso no), podemos inferir que la siguiente evaluación sea de otra Y.
%	En este caso, Y sería Y -$>$ \^\ T, y este T sreía T -$>$ num Y, del cual Y -$>$ e.
	
%	De esta forma concluímos la derivación con (num + num) \^\ num
	
%	Ya teniendo la expresión numérica evaluada, terminamos de evaluar el print con el siguiente token que es ')'.
%	Luego, para finalizar, revisamos que la instrucción termine con un ';', el cual en este caso existe y aparece en el final de la expresión.
	
%\subsection*{Semántica}
%	Una vez definido el árbol de sintáxis abstracta dado por el parser, pasamos a evaluar la semántica.
	
%	Definimos una clase abstracta que Expression será la madre de todas nuestras diferentes expresiones:
%\begin{itemize}
%	\item Toda expresión posee un tipo y una instancia de \emph{Error Semántico}.
%	\item Las expresiones tienen que poder validarse y, luego de ser válidas, evaluarse para devolver el valor que expresan.
%\end{itemize}

%	Sigamos con el ejemplo de la vez anterior, tiendo un árbol de derivación resultante de esta forma:
	
%	\begin{center}
%			$ $ print $ $ \\
%	$ $ $ $ $ $ $|$ $ $ $ $ \\
%		$ $ $ $ exp $ $ \\
%	$ $ $|$ $ $ $ $ $|$ $ $ \\
%	sum $ $ 3 $ $ $ $ \\
%	$|$ $ $ $|$ $ $ $ $ $ $ $ $ $ $\\
%	1 $ $ $ $ 2 $ $ $ $ $ $ $ $ $ $\\
%	\end{center}
	
%	Empezamos a validar desde afuera hacia adentro, aunque luego la ejecución terminará validando primero las expresiones más internas para que las externas puedan ser validadas basándose en ellas.
	
%	Comenzando por el print, solo debemos validar la expresión que lleva en su interior.
%	Dentro tenemos la expresión numérica compuesta por una expresión binaria de exponente de: una expresión binaria de suma entre los literales numéricos '1' y '2', y el literal numérico '3'.
%	Los literales numéricos son válidos de por sí, así que no hace falta evaluarlos (al hacerlo siempre devuelve que son válidos), así que pasamos a evaluar las expresiones binarias.
	
%	Al validar nos daremos cuenta que la instrucción es válida, no aparecen llamadas a métodos ni a variables que es de lo que mayormente se trata de verificar su validez, apoyándose en los contextos para guardar las variables y las definicionesde las funciones.
	
%	Al ser esta instrucción válida, es momento de evaluar para definir el valor que va a ser devuelto por esta instrucción, ya que en este caso devuelve un valor.
%	Comenzamos igualmente desde la expresión más externa hasta llegar a la más interna y devolver así todo el valor calculado.
%	Primero llegamos a evaluar la expresión binaria de la suma de '1' y '2', esto devuelve '3'.
%	Luego esto se le devuelve al miembro izquierdo de la expresión binaria de exponente de, ahora '3', y '3', que devuelve '27'.
%	Por último se devuelve el '27' al print que lo devuelve igualmente para que salga en pantalla para el usuario, concluyendo así tanto el proceso de evaluar el valor, como el proceso de compilación en general.
\end{document}